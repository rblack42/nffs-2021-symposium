\subsection{Generating STL Files}

Many folks who use \osc\ generate {\it STL} files from their \osc\ designs. These
files are used as part of the 3D printing process. Although I wish we could 3D
print flyable model airplanes, that is not what we will do with the STL files
in this project.

STL stands for {\it Standard Tessellation Library} \citep{stlwiki}. Translated,
that means an STL file is a list of triangular objects that describe the
surface of a 3D object. These triangles cover the surface making a ``water
tight'' approximation to the real surface. Triangles are guaranteed to be flat
in the 3D space, and if they are tiny enough, they can be rendered to create a
3D display of the object, or sent to a 3D printer after suitable processing of all
the triangles so the printer knows where the surface is located.

My purpose in introducing these triangles is simple. I want to know how much my
model will weigh, and I want to know where the center of gravity will be when
the design is constructed. Figuring out these two details is impossible using
conventional building techniques: you build the model, weight it, and figure
out the center of gravity manually. Computer geeks never do anything manually
if they can get their computer to do the work.

I found some nice \PY\ code that is all set up to figure out the volume of a 3D
shape defined in an STL file, and return it's center of gravity location. It is
simple enough to add in the predicted wood density for the design and come up
with a weight estimate for each part. A little post-processing of all this data
will give us some estimate of the total designs weight. Unfortunately, figuring
out the weight of the glue is not so simple. For that we need to know where
each glue joint will be and the surface area of the glue joint. Fortunately, by
designing our model using separate parts, this turns out to be fairly easy to
do!

Knowing where each part will be placed in the final design, and the CG data for
that part will let us predict the CG for the complete model as well!

Let's demonstrate this idea using a simple rib section.

\subsubsection{Estimating Weights}

I have worked to set up each major part you would need to create out of balsa,
or other materials as a module we can activate separately. In this example, I
will show how to estimate the weight of a single wing rib.

For this section, I will be using \osc\ as a {\it command-line} tool,
something that may not be familiar to non-programmers. Details on this are on
the project website, but basically, we will open up a window where we can type
in commands to the operating system. This will involve running a {\bf cmd.exe}
on a Windows system, and opening up a {\it terminal} on Mac and Linux systems.

We will run \osc\ as a tool and will not open up the graphical user
interface you normally see. In this mode, \osc\ will read a file, process it
and generate an output file containing the results we want. In this example, we
will be asking it to generate an STL file from a part definition {\it scad}
file.

My {\bf rib.scad} file is set up to display a single rib if you load it in
OpenSCAD normally. The command I give to my Operating System to generate the
{\it STL} file looks like this:

\begin{lstlisting}
openscad rib.scad -o rib.stl
\end{lstlisting}

That is not so bad. If you check the directory where we ran this command (the
one containing {\bf rib.scad}, you will see a new file named {\bf rib.stl}. You
can examine this file, but all it contains is a bunch of definitions of the
triangles needed to enclose a single rib.

Using the Python library to get the rib data is pretty easy.  created a simple
\PY\ program to process this STL file. The program is named {\bf getvolume.py}.
I will not show the code here, but it is available on the project website.

This program is run from the command line as follows:

\begin{lstlisting}
python getvolume.py rib.stl 4.5
\end{lstlisting}

This line tells the operating to run \PY\ using my program code file. The next
two parameters are the name of the part STL file followed by density of the part
in pounds per cubit foot.

This will process the rib.scad file in the current directory, and print out the
volume and CG information. It also prints the bounds of the box occupied by the
shape, and the expected weight for the part.

Here is the output I saw from my test:

\begin{lstlisting}
Processing rib with density 4.5
Shape bounds are:
  X: 0.0 <-> 5.0
  Y: -0.0156 <-> 0.0156
  Z: 0 <-> 0.362
Shape volume: 0.0099 cubic inches
Shape CG    : [2.5000 0.0000 2.2725e-1]
Estimated weight = 0.01157 grams
\end{lstlisting}

That is pretty cool, and the numbers look reasonable for this rib!

\subsubsection{Glue Weight}

We still need to figure out how much glue we will need. The scheme I came up
with uses the \osc\ i{\it intersection} operation to give is an estimate of
the glue surface.

It works like this. We take the two parts we want to glue together and slide one of them a tiny bit into the other part. Using the {\it intersection} operation will leave us with a tiny surface part that we can use as a model for the glue we will use. Of course, if you are a sloppy builder like me, this amount of c=glue might be off, but it is still a good way to get a weight estimate for the glue we will use.

I found some data on glue weights in a n article authored by Larry Coslick that I am using until I get some data of my own.


