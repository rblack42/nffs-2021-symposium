\subsection{Bottom-up Design}

We could work through our design in a ``top-down'' manner, breaking up the model
into major assemblies, then breaking down those assemblies into individual parts
that need to be manufactured to complete the real model. However, using \osc\ we
cannot see much until we have parts  to build with. For that reason, we will
begin with simple parts, then assemble them by properly positioning things to
create assemblies. In the final step, we will put everything together so we can
see our completed model! This all should make sense to an indoor model builder.
You cut out the pieces you need, glue things together to form wings and
stabilizers for instance, then assemble the final model using something like
paper tubes perhaps.

We will start off by writing down the constraints we must obey to compete in the
\LPP\ class. I create a data file for this information, then add various other
dimensions as I make design decisions. I can {\it include} this data file in any
other file that needs this data.

As soon as we have enough information in hand to define the basic parts, we will
begin building our CAD models. This will involve writing some simple code
describing individual parts. Along the way, we will be defining other parameters
like material thickness and overall parts dimensions.  As much as possible
dimensions will be derived from other data parameters defined in the design. For
example, the rules constrain the maximum chord for the wing. Once i know how
thick my leading and trailing edges will be, I can calculate the chord of the
rib I will place between those two parts. The constraints drive the process, as
much as possible. We will document each design decision we make in the form of
parameters, or math equations that are driven by other parameters in our design.

\subsubsection{Parametric designs}

Setting up the design in this way lets us adapt this design to create other
designs. We might need to change the overall wingspan of the model, to compete
in another class. We should be able to modify our constraining parameters, and
let \osc generate a new model for the new rules. This is far better than simply
scaling a PDF file so the wingspan is different.

A further benefit of this process is that you no longer confront plans that
leave out critical dimensions. I do not like trying to figure out how to build
some part by taking my digital calipers and measure things on a plan, or worse,
a  PDF printout, then trying to figure out the real dimensions! \osc forces you
to completely define your model.

