\subsection{Installing the tools}

You will need to install two free programs to follow along with this design: \osc\ and \PY.

\subsubsection{Installing \osc}

Installing \osc\ is pretty simple using instructions found on the program's 
website \cite{openscad}.
Basically, you download the installer for your system then run that program. Once it is installed, open it up and take a look at the basic
interface.  Figure \ref{fig:opening-screen.jpg} is what you should see.

\importimage{opening-screen.jpg}{\osc\ interface}

There are three areas we will be using in this view:

\begin{itemize}
\item{Editor - the left panel where you type in your code.}
\item{Preview - the top right panel will be where your model is displayed.}
\item{Messages - the bottom right panel is where error messages will be shown
when processing your code.}
\end{itemize}

I will not attempt to show everything you need to know about the language \osc\
uses for describing a model. Instead, I will show fragments of code to give you
a feel for what you need to write to design your model. The project
website~\cite{blackr} has more details, as does the \osc\ {\it User
Manual}~\cite{userman}.

I highly recommend that you  print out a copy of the \osc\
``cheat-sheet'' is available here:~\cite{osccheat}. It will be a good reference
as you see \osc\ code examples.

\subsubsection{Installing Python}

In the analysis part of this discussion, we will be using some \PY\ programs to
do some of our work. \PY\ is another free tool, available for all platforms. It
even comes pre-installed on some (sadly, not on PCs though). You should install
this tool if you wish to follow along with this design. \PY\ is really only
needed for the analysis phase of your design work.

Navigate to the \PY\ website \cite{python} and download an installer for your
system. \PY\ does come with a simple editing tool, and there are many nicer
development tools available. We will not explore those tools here. More details
are on the \MM\ project website.

\subsubsection{Command Line}

Finally, some of the concepts in this article require running programs from the
{\it command line}. Many of you have never seen this interface, although it is
available on all systems. Basically, instead of clicking on some icon with your
mouse to start up a program, you enter a line of text into the interface and
tell the operating system what you want to do. I will only show a little of this
here, more details are on the project website. You get to this interface by
entering {\bf cmd} in the Windows search box, or my opening the {\it Terminal}
program on  Mac systems. Linux users will open up a {\it shell} command window.
We will only need to use this interface in the analysis of the design.


Ready to get started? Let's look at \osc!


