\section{Introduction}

In case you have not noticed, 3D printers are becoming as common as the X-Acto
knife in home shops these days. They let you, or your kids, build some amazing
parts, some of which find their way into our model airplanes. If you look
around on {\it Thingiverse} \cite{thingy} where a lot of these 3d models can be
found, you will discover that many of the files needed for 3D printing were
generated using a neat open-source {\it Computer Aided Design} tool called \osc.
I have used this tool for a number of projects, and I decided to see how it
could help with my current indoor model building. In this article I will show
how I developed a new design for a \LPP\ model using \osc. I named the design
\MM\ since I used a fair amount of math in its design.

\osc\ is not your usual CAD tool. It is a programmer's tool, meaning that you
write fairly simple program code that describes your model. Then \osc\ processes
that code to generate a visual representation of your design that you can see on
your computer screen. Once you are happy with the design, \osc\ can export your
design in file formats that could be post-processed as part of the path to a 3D
printer. Unfortunately, we probably are not going to see competition ready 3D
printed indoor models anytime soon, so we will not explore 3D printing in this
article.

Do not be frightened off by having to write program code for \osc. Like
anything new, it can be a bit intimidating for beginners, but I have worked on
making the code needed simple enough that even non-technical folks should be
able to generate useful results.

This article is organized in three major sections. In {\it Part 1} I will
explain the basic concepts used in \osc\ to set up a 3D design. This is not a
complete tutorial, but covers enough to help you understand the rest of the
article,

In {\it Part 2} we will work through the design on my \LPP\ model. You will see
some code here, but only enough to explain how I generated the design. Space
limitations in this article prevent showing everything you need. All of the
code and more detailed documentation on this project is available on the
project website at {\bf https://rblack42.github.io/math-magik} \cite{mmagik}.

Finally, {\it Part 3} will show some analysis techniques I used to get a weight
and balance assessment of my design.  This is always a good thing to do before
going flying! In this section, I will show a bit of Python code. \PY\ is
another programming language, commonly used in introductory programming
courses. Kids in elementary school have managed to get going in programming
using Python.






