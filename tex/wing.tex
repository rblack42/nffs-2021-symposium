\section*{Building the Wing}

Let's start off this design by building the wing.

\subsection*{Design Constraints}

The class rules specify two constraints on the design of the wing. The
projected span is limited to 18 inches, nd the maximum chord is limited to five
inches. Based on a survey of indoor models, we will use a flat center section
with wing tips angled upward, providing the needed dihedral for stability. Here
is our starting geometry:

\importtikzfigure{wing-geometry}{Basic Wing Layout}


The constraint in this figure is {\bf max\_span}. We are free to choose the
size of either the dihedral angle, or offset of the tip section.

\subsection*{Circular Arc Airfoils}

Many indoor model airplanes use a simple circular arc airfoil.

Usually, these models have some specified chord for either the wing or
stabilizer, and the airfoil is specified as an arc with a maximum height that
is some percentage of that chord. So, we are given the chord, and thickness as
a percentage value. The challenge is to figure out the radius of the arc that
satisfies these values.

\subsubsection*{Arc Geometry}

Since we start off with the chord and thickness specified as a percentage, we
need to get rid of the percentage:

Given:

\begin{itemize}
  \item{{\bf c} - the chord of the wing}
  \item{{\bf T} - the camber as a percentage of {\bf c}}
\end{itemize}

\begin{equation}
    t =  T c / 100
\end{equation}

Here is a diagram showing what we are dealing with:

\importtikzfigure{circular-arc}{Circular arc Geometry}

From this figure, we can write two equations:

\begin{equation}
  {r \sin(\theta) = \frac{c}{2}}
\end{equation}

\begin{equation}
  {r - r \cos(\theta) = t}
\end{equation}

The unknown variables are now:

\begin{itemize}
  \item{{\bf r} - the radius of the circular arc}
  \item{{$\theta$} - one half of the total angle swept by the arc}
\end{itemize}

While we could drag out our old math books from high school, I would rather let
my computer do the hard work. Time for SymPy \cite{sympy}!

\subsubsection{SymPy}

SymPy is a Python program that knows how to do a lot of math. It manipulates
{\it symbols}, not numbers, and that is a lot of what you should have learned in
a trigonometry classes.

Here is how we get SymPy to solve our equations:

\includecode{python/circular-arc.py}

Running this code, we find our solution:

\begin{sympyblock}
r,c,t,theta = sympy.symbols('r c t theta')

eq1 = 2 * r * sympy.cos(theta) - c
eq2 = r - r * sympy.sin(theta) - t
sol = sympy.solve([eq1,eq2],[r,theta])

\end{sympyblock}

Here are the solutions:
\begin{equation}
\sympy{sol}
\end{equation}

Here, I asked SymPy to solve the two equations for the two {\it symbols} {\bf r}
and {$\theta$}. The two solutions are shown. The first one is the {\bf radius} we need to our code.  We do not really need to worry
about the angle part.

We will use these result to set up an OpenSCAD module that will build our ribs.

Warning! If there are any parents reading this, do not let your kids know about
SymPy.  On the other hand, if you cannot solve your kids math homework, you
might try SymPy on the sly! YMMV!

\subsection{Ribs}

The ribs for this model will be simple circular arcs. We need to specify the
chord, and the camber thickness as a percentage of the chord, we also need to
specify how thick the rib will be and how deep the rib will be. We will create
a simple module to let the user generate as many ribs as needed, adjusting each
with the needed parameters:

\begin{lstlisting}
rib(
  chord=5,
  camber=6,
  thickness=1/32,
  height=1/16
);
\end{lstlisting}

When I defined the rib module, I set un default values as shown above, so to
activate a standard rib, I just need to write {\bf rib()}.

The code needed to generate a rib is a bit involved, but basically involves
creating a short cylinder, cutting out the center of that cylinder leaving a
tube, then slicing off the part of the ring that leaves just the rib. We then
rotate it and align it for use later.

Here is an example rib, ready for use:

\importimage{basic-rib.png}{Basic rib}

\subsection{Wing Center Section}

The leading and trailing edges of this model will be simple 1/16" balsa sticks.
These are easy to model oin OpenSCAD as really skinny cubes. The center section
of the wing holds five equally spaced ribs. This code is pretty simple, so I
packaged it in a simple module:

\begin{lstlisting}
wing_center_section(
  chord=5, nribs= 5, span=12
);
\end{lstlisting}

Here is this code:

%\includecode{scad/wing_center.scad}

This code uses a loop to place the ribs. Details on all of these language
constructs can be found in the {\it Users Manual} \ref{scadman}

Figure \ref{fig:wing-center.png}{Wing Center Section}.

\importimage{wing-center.png}{Wing Center Section}

There is one detail in this image worth examining closer. Figure
\ref{fig:wing-tip} shows the joint at the outer rib. This allows the tip section
to mate at this joint.

\importimage{tip-joint.png}{Tip Joint Detail}

Being able to zoom in on details like this is handy for making sure your design
is clean.

\subsection*{Tip Design}

The tip is another rectangular section, except I decided to round off the
leading edge corner. There are no ribs in this section, except for a flat rib
at the tip. Here is the geometry I decided on:

\importtikzfigure{wing-tip}{Tip Design}

The tip module uses a simple supporting module that generates the curved section.
This will be a balsa stick formed over a template.  By writing the tip as a
module, I will be able to reuse this design on the stabilizer and the fin!

\begin{lstlisting}
tip_section(span=3, chord=5, radius=2);
\end{lstlisting}

There is one interesting issue in the tip design. Many builders taper the
leading and trailing edges so the tip is lighter. In this design, I decided to
taper just the leading and trailing edge tip spars and make the circular arc
and the tip thinner square stock. The puzzle is how to do this in OpenSCAD.

It turns out to be easy. Once again, we use the {\it difference} operation. We
generate a square strip the size we want, then set up a block longer than the
strip and place it at a slight angle so we end up chopping off the unwanted
material leaving a tapered strip. It is easy enough to do this twice to get the
result we are after. This is sanding with no mess!

Figure \ref{fig:trim-block.png}  an image showing the basic idea

\importimage{trim-block.png}{Trimming Block}


FIgure{wing-tip.png}{Wing Tip} shows the final tip section.

\importimage{wing-tip.png}{Wing Tip}

\subsection{Tip Templates}

I kept this design simple. The only templates needed will be used for forming
the curved segments at the tips. Fortunately, OpenSCAD can help generate
printed templates we can use!

Basically, we set up code that generates a tip section, then flatten it back
into a 2D drawing. The operation that does this is called a {\bf projection}.
Once that is set up, we will ask OpenSCAD to export this model as an SVG file
which can be printed. I use my Chrome browser to do the printing on my home
printer.  Figure \ref{tip-template.png} shows what the template looks like in
Chrome.

\importimage{tip-template}{Tip Template}

\subsection{Wing Assembly}

All we need to to to complete the wing is generate the center section, then generate the wing tips and rotate them into place at the proper dihedral angle. The only thing tricky here is generating the two tip sections.

Since the tip is flat, I can generate two of then, and rotate one 180 degrees so it will fit on the opposite side. I do need to do some minor translating to make sure things line up, but by now, this is getting easy!

Figure{fig:wing.png}{Final Wing} shows the completed wing.

\importimage{wing.png}{Wing Assembly}
