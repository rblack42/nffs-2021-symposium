\subsection{Propeller}

Now for an interesting component - the propeller. This is not just a flat part,
it has an interesting shape, and figuring out how to model this thing took some
time. In the end, I remembered how most of us build indoor propellers. We take a
thin sheet of balsa, cut out the blade profile we want, then soak that blade
for a while. Next, we tape it to the side of a round can at an angle and bake
it. When it dries, it has a curvature that will work fairly well.

It struck me that I could generate the same shape by creating a polygon that
represents the blade outline, then extrude that to form a very thick blade
(bigger that the can!) I then build a hollow cylinder with a thickness that
matches our desired blade thickness and slide the extruded blade into that cylinder.
The {\it intersection} of these two shapes will leave us with a curved blade.
Neat!

The shape of the blade seems to be a matter of taste. Many builders design
blades that will provide for a flair so that the prop will have a higher pitch
when the plane is launched with full torque from the motor. Moving the prop
spar toward the trailing edge of the blade provides this flair.

I decided to generate a simple blade layout, with parameters that can be
adjusted to give the flair you want. Figure \ref{fig:prop-blade} shows the
general layout.

\importskinnytikzfigure{prop-blade}{Prop Blade Layout}

The prop spar is a tapered cylinder, something easy to generate with the {\it
cylinder} shape. Parameters are provided so the size of the spar can be
adjusted.
i
Figure \ref{prop-blade.png} shows the final blade produced. Since the code is
again a bot complicated, I will refer you t the project website.

\importimage{prop-blade.png}{Prop Blade}


