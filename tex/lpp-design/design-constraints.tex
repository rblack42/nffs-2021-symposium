\section{Design Constraints}

The {\it Limited Pennyplane} class rules define a few constraints on dimensions
for our model. Specifically we must honor these limits:

Here is the start of my data file, where the important class dimensions are
defined:

\includecode{scad/math-magik-data.scad}

What these constraints mean is that the model must fit in a box that measures {\bf
max\_wing\_span} wide by {\bf max\_length} long. There is no limit on how
tall this box can be.

Furthermore, the wing must fit in a smaller box measuring {\bf max\_wing\_span}
by {\bf max\_wing\_chord}. The stabilizer must fit in a similar box measuring
{\bf max\_stab\_span} by {\bf max\_stab\_chord}. There are no constraints on where
these boxes fit inside the outer box. The propeller is only limited by
diameter, blade shape it up to the designer. However, the {\bf max\_length}
constraint is measured from the forward-most point, usually on the propeller,
to the aft-most point on the model. We could build a pusher, but I have not
considered that idea.

\importtikzfigure{lpp-design-constraints}{LPP Design Constraints}

Note: The labels in this diagram are abbreviations for the names shown above.
In my code I will use full names to improve readability of the code.
