\section{Building the Wing}

Let's start off this design by building the wing.

\subsection*{Basic Wing Design}

The class rules specify two constraints on the design of the wing. The projected
span is limited to 18 inches, and the maximum chord is limited to five inches.
Based on a survey of indoor models, we will use a flat center section with wing
tips angled upward, providing the needed dihedral for stability. We can define
either the tip angle we want or the elevation of the tip. In either case, we can
calculate the other item in our code. 

Here is our starting geometry:

\importtikzfigure{wing-dihedral}{Basic Wing Dihedral Layout}

\subsection{Circular Arc Airfoils}

Like many indoor model airplanes, we will use a simple circular arc airfoil.
The airfoil is typically specified as an arc with a maximum height that is some
percentage of the wing chord.  So, given the chord, and thickness as a
percentage value, we need to figure out the radius of the arc given these two
values.

\subsubsection{Arc Geometry}

Since we start off with the chord and thickness specified as a percentage, we
need to get rid of the percentage:

Given:

\begin{itemize}
  \item{{\bf c} - the chord of the wing}
  \item{{\bf T} - the camber as a percentage of {\bf c}}
\end{itemize}

\begin{equation}
    t =  T c / 100
\end{equation}

Here is a diagram showing what we are dealing with:

\importtikzfigure{circular-arc}{Circular arc Geometry}

From this figure, we can write two equations:

\begin{equation}
  {r \sin(\theta) = \frac{c}{2}}
\end{equation}

\begin{equation}
  {r - r \cos(\theta) \end{equation}

The unknown variables are now:

\begin{itemize}
  \item{{\bf r} - the radius of the circular arc}
  \item{{$ \theta $} - one half of the total angle swept by the arc}
\end{itemize}

While we could drag out our old math books from high school, I would rather let
my computer do the hard work. I did this mat work using a cool \PY\ package
called {\it SymPy}~\cite{sympy}! Details on this are at the project website.

Here are the solutons:

\begin{sympysilent}
r,c,t,theta = \
  sympy.symbols('r c t theta')

eq1 = 2 * r * sympy.cos(theta) - c
eq2 = r - r * sympy.sin(theta) - t
sol = sympy.solve([eq1,eq2],[r,theta])
\end{sympysilent}

\begin{equation}
\sympy{sol}
\end{equation}

The first solution is the {\bf radius} we need to our code. The second one gives
the half-angle swept by the arc for our rib. We do not really need to worry
about that angle. 

\subsection{Wing Thickness function}

In order to meet one of the design goals for this project, we need to come up
with a formula that will give us the height of the wing at any point . Since we
are uging a circular arc airfoil, these equations get reasonably easy to figure
out. (If you have trouble following this math, ask your kid for help, or your
grandkid!)

\importtikzfigure{wing-thickness}{Wing Thickness Geometry}

From figure \ref{wing-thickness} we can set up a few more equations.


Suppose we are interested in the height of our airfoil at any point along the
{$x$} axis. From basic geometry, we come up with this equation

\begin{equation}
  h = r cos{\beta} - ( r - t )
\end{equation}

We figured out the value for {$r$} in our earlier math work.


To figure out {$\beta$} we need to use the $\arcsin$ function.

\begin{equation}
  x = C / 2 - r \sin(beta)
\end{equation}

Rearranging things we get:

\begin{equation}
  \beta = \sin^{-1} ( \frac{x - C/2}{r} )
\end{equation}

That equation will give us the height of the wing at any point, depending on
the camber, the local chord, and the value of {$x$} we use. The remaining
problem is determining the chord at any point along the span of the wing.

Figure \ref{wing-geometry} shows the general layout of the wing and stab. (We
can also use this for the fin, with minor tweaks.)

\importtikzfigure{wing-geometry}{General Wing Geometry}

The wing will have a nice constant airfoil across the inner portion. At the dihedral break, the height will taper off until it is flat at th tip. I will assume we can use a constantly decreasing camber across the tip, running from the inner airfoil value to zero at the tip. The functin we need to calculate the chord and camber at any point along the span liiks like this:

The general equation of or a circle centered at point {$c$} is:

\begin{equation}
  {(x - c_x)}^2 + {(y - c_y)}^2 = r^2
\end{equation}

This is simple enough to solve for {$y$}, but I did let {\it sympy} chew on it just to check:

\begin{equation}
  y = c_y - \sqrt{r^2 - (x - c_x)^{2}}
\end{equation}

These equations are being used in an OpenSCAD function when we need to figure out how high the wing is at any point.



%\newcommand{\threepartdef}[6] {
	%\left\{
		  %\begin{array}{lll}
			  %#1 & \mbox{if } #2 \\
			  %#3 & \mbox{if } #4 \\
			  %#5 & \mbox{if } #6
		  %\end{array}
	  %\right.
%}

%/begin{equation}
%$\mu(n)$ = \threepartdef
%{1}      {n=1}
%{0}      {a^2 ,|, n \mbox{ for some } a > 1}
%{$(-1)^r$} {$n$ \mbox{ has } r \mbox{ distinct prime factors}}
%\end{equation}


Now that we have our equations figured out, we can proceed with building a
module that will generate a single rib. We will write the rib module so it
takes four parameters:

\begin{lstlisting}
module rib(
    chord = 5,
    camber = 6,
    height = 1/16,
    thickness = 1/32) {
  ...
}
\end{lstlisting}

\osc\ supports {\it global variables}. This means you
can define a variable at the top of a file,  give it a value, then use
that variable anywhere in the code. Unfortunately, that makes it hard to take
your module to another program, since it depends on those variables. Writing
your modules so they depend only on data they receive through parameters makes
for better code.

The {\bf rib} module's job will be to generate a single rib sitting on the {bf X} axis
with it's nose at the origin, and standing upright.

Notice that I defined  default values for each of the parameters, If you are
happy with those, you can generate a rib by writing {\bf rib();}. You can
change any of those by providing the parameter name, an equal sign, and a new
value. So generating a rib with only a new chord value would look like this:
{\bf rib(chord=4);}. All other parameter default values still apply.

The code needed to generate a rib is a bit involved, so I will not show it
here. Basically, it creates a cylinder using the radius we calculated from our
formula, and a height equal to the rib thickness.  Then it cuts out the center
of that cylinder leaving a ring that will have the required rib stick height.
Next, it slices off the unneeded part of the ring leaving just the rib. The
final step rotates it upright, and aligns on the {\bf X} axis, ready for use!

Here is an example rib:

\importimage{basic-rib.png}{Basic rib}

\subsection{Wing Center Section}

The leading and trailing edges of this model will be simple 1/16" balsa sticks.
These are easy to model in \osc\ as really skinny cubes. The center section
of the wing holds five equally spaced ribs. This code is pretty simple, so I
packaged it in a simple module:

Here is this code:

\includecode{scad/wing_center.scad}


This code uses a loop to place the ribs. Details on all of these language
constructs can be found in the {\it Users Manual} \ref{userman}

Figure \ref{fig:wing-center.png} shows the wing center section.

\importimage{wing-center.png}{Wing Center Section}

There is one detail in this image worth examining closer. Figure
\ref{fig:wing-tip} shows the joint at the outer rib. This allows the tip section
to mate at this joint.

\importimage{tip-joint.png}{Tip Joint Detail}

Being able to zoom in on details like this is handy for making sure your design
is clean.

\subsection{Tip Design}

The tip is another rectangular section, except I decided to round off the
leading edge corner. There are no ribs in this section, except for a flat rib
at the tip. Here is the geometry I decided on:

\importtikzfigure{wing-tip}{Tip Design}

The tip module uses a simple supporting module that generates the curved section.
This will be a balsa stick formed over a template.  By writing the tip as a
module, I will be able to reuse this design on the stabilizer and the fin!

\begin{lstlisting}
tip_section(span=3, chord=5, radius=2);
\end{lstlisting}

There is one interesting issue in the tip design. Many builders taper the
leading and trailing edges so the tip is lighter. In this design, I decided to
taper just the leading and trailing edge tip spars and make the circular arc
and the tip thinner square stock. The puzzle is how to do this in OpenSCAD.

It turns out to be easy. Once again, we use the {\it difference} operation. We
generate a square strip the size we want, then set up a block longer than the
strip and place it at a slight angle so we end up chopping off the unwanted
material leaving a tapered strip. It is easy enough to do this twice to get the
result we are after. This is sanding with no mess!

Figure \ref{fig:trim-block.png} shows the basic idea.

\importimage{trim-block.png}{Trimming Block}


Figure \ref{fig:wing-tip.png} shows the final tip section.

\importimage{wing-tip.png}{Wing Tip}

\subsection{Tip Templates}

I kept this design simple. The only templates needed will be used for forming
the curved segments at the tips. Fortunately, \osc\ can help generate
printed templates we can use! Here is the code used.

\includecode{demo/printer-test.scad}

The {\it scale} operation is what sized the shape so it prints accurately. Most
of the rest of the code just made adjustments so the template printed properly
on my printer. These might need adjusting for different printers.

Basically, we set up code that generates a tip section, then flatten it back
into a 2D drawing. The operation that does this is called a {\it projection}.
Once that is set up, we will ask \osc\ to export this model as an SVG file
which can be printed. I use my Chrome browser to do the printing on my home
printer.  Figure \ref{fig:tip-template.png} shows what the template looks like
in Chrome.

\importimage{tip-template}{Tip Template}

\subsection{Wing Assembly}

All we need to to to complete the wing is generate the center section, then
generate the wing tips and rotate them into place at the proper dihedral angle.
The only thing tricky here is generating the two tip sections.

Since the tip is flat, I can generate two of then, and rotate one 180 degrees
so it will fit on the opposite side. I do need to do some minor translating to
make sure things line up, but by now, this is getting easy!

Figure{fig:wing.png}{Final Wing} shows the completed wing.

\importimage{wing.png}{Wing Assembly}
