\subsection{Building the Wing}

Let's start off this design by building the wing.

\subsubsection*{Basic Wing Design}

The class rules specify two constraints on the design of the wing. The projected
span is limited to 18 inches, and the maximum chord is limited to five inches.
Based on a survey of indoor models, we will use a flat, rectangular center section with wing
tips angled upward, providing the needed dihedral for stability. We can define
either the tip angle we want or the elevation of the tip. In either case, we can
calculate the other item in our code.

\importtikzfigure{wing-geometry}{General Wing Geometry}

Figure \ref{fig:wing-geometry} shows the general layout of the wing and stab. (We
can also use this for the fin, with minor tweaks.)


The tip uses another simple circular arc for the leading edge.

Figure \ref{fig:wing-dihedral} shows the dihedral configuration.

\importtikzfigure{wing-dihedral}{Basic Wing Dihedral Layout}

\subsubsection{Circular Arc Airfoils}

Like many indoor model airplanes, we will use a simple circular arc airfoil.
The airfoil is typically specified as an arc with a maximum height that is some
percentage of the wing chord.  So, given the chord, and thickness as a
percentage value, we need to figure out the radius of the arc given these two
values.

Since we start off with the chord and thickness specified as a percentage of the
chord, we need to get rid of the percentage:

Given:

\begin{itemize}
  \item{{\bf c} - the chord of the wing}
  \item{{\bf T} - the camber as a percentage of {\bf c}}
\end{itemize}

\begin{equation}
    t =  T c / 100
\end{equation}

Here is a diagram showing what we are dealing with:

\importtikzfigure{circular-arc}{Circular arc Geometry}

From this figure, we can write two equations:

\begin{equation}
  {r \sin(\theta) = \frac{c}{2}}
\end{equation}

\begin{equation}
  t = r - r \cos(\theta)
\end{equation}

The unknown variables are now:

\begin{itemize}
  \item{{\bf r} - the radius of the circular arc}
  \item{{$ \theta $} - one half of the total angle swept by the arc}
\end{itemize}

While we could drag out our old math books from high school, I would rather let
my computer do the hard work. I did this math work using a cool \PY\ package
called {\it SymPy}~\cite{sympy}! This program manipulates math equations as
symbols, like you used back in high school. Details on this are at the project
website.

Here are the solutions:

\begin{sympysilent}
r,c,t,theta = \
  sympy.symbols('r c t theta')

eq1 = 2 * r * sympy.cos(theta) - c
eq2 = r - r * sympy.sin(theta) - t
sol = sympy.solve([eq1,eq2],[r,theta])
\end{sympysilent}

\begin{equation}
\sympy{sol}
\end{equation}

The first solution is the {\bf radius} we need to our code. The second one gives
the half-angle swept by the arc for our rib. We do not really need to worry
about that angle.

\subsubsection{Wing Thickness function}

In order to meet one of the design goals for this project, we need to come up
with a formula that will give us the height of the wing at any point . Since we
are using a circular arc airfoil, these equations get reasonably easy to figure
out.

\importtikzfigure{wing-thickness}{Wing Thickness Geometry}

From figure \ref{fig:wing-thickness} we can set up a few more equations.


Suppose we are interested in the height of our airfoil at any point along the
{$x$} axis. From basic geometry, we come up with this equation

\begin{equation}
  h = r cos{\beta} - ( r - t )
\end{equation}

We figured out the value for {$r$} in our earlier math work.


To figure out {$\beta$} we need to use the $\arcsin$ function.

\begin{equation}
  x = C / 2 - r \sin(beta)
\end{equation}

Rearranging things we get:

\begin{equation}
  \beta = \sin^{-1} ( \frac{x - C/2}{r} )
\end{equation}

That equation will give us the height of the wing at any point, depending on
the camber, the local chord, and the value of {$x$} we use. The remaining
problem is determining the chord at any point along the span of the wing.

The wing will have a nice constant airfoil across the inner portion. At the
dihedral break, the height will taper off until it is flat at the tip. I will
assume we can use a constantly decreasing camber across the tip, running from
the center section airfoil to a flat tip. We need to come up with an equation
to calculate the chord

The tip is a circular arc. From that we can use the general equation of or a
circle centered at point {$c$} is:

\begin{equation}
  {(x - c_x)}^2 + {(y - c_y)}^2 = r^2
\end{equation}
 bit of rearranging gives us this result:

\begin{equation}
  y = c_y - \sqrt{r^2 - (x - c_x)^{2}}
\end{equation}

Using the height equation, and the chord equation, we have everything we need
to figure out the height of the wing surface at any point. We will need  that
later.

\subsubsection{The Rib Module}

Now that we have our equations figured out, we can proceed with building a
module that will generate a single rib. We will write the rib module so it
takes four parameters to make using it for different chords easy,

\begin{lstlisting}
module rib(
    chord = 5,
    camber = 6,
    rib_height = 1/16,
    rib_thickness = 1/32) {
  ...
}
\end{lstlisting}

I omitted the details here.

The {\bf rib} module's job will be to generate a single rib sitting on the {bf X} axis
with it's nose at the origin, and standing upright.

Notice that I defined  default values for each of the parameters, If you are
happy with those, you can generate a rib by writing {\bf rib();}. You can
change any of those by providing the parameter name, an equal sign, and a new
value. So generating a rib with only a new chord value would look like this:
{\bf rib(chord=4);}. All other parameter default values still apply.

The code needed to generate a rib is a bit involved, so I will not show it
here. Basically, it creates a cylinder using the radius we calculated from our
formula, and a height equal to the rib thickness.  Then it cuts out the center
of that cylinder leaving a ring that will have the required rib stick height.
Next, it slices off the unneeded part of the ring leaving just the rib. The
final step rotates it upright, and aligns on the {\bf X} axis, ready for use!

Here is an example rib:

\importimage{basic-rib.png}{Basic rib}

\subsubsection{Wing Center Section}

The leading and trailing edges of this model will be simple 1/16" balsa sticks.
These are easy to model in \osc\ as really skinny cubes. The center section of
the wing holds five equally spaced ribs. This code is pretty simple, so in the
interest of keeping this article reasonable short, I will not show it here. All
the code is placed in a simple {\bf wing-center} module. Basically al it does
is position the leading and trailing edge spars, then creates the five ribs,
translating each into proper position. Figure \ref{fig:wing-center.png} shows the
result.

\importimage{wing-center.png}{Wing Center Section}

\subsubsection{Tip Design}

The tip is another rectangular section, except I decided to round off the
leading edge corner. There are no ribs in this section, except for a flat rib
at the tip. Here is the geometry I decided on:

\importtikzfigure{wing-tip}{Tip Design}

The tip module uses a simple supporting module that generates the curved section.
This will be a balsa stick formed over a template.

\begin{lstlisting}
wing_right tip(
    span=3, 
    chord=5, 
    radius=2
);
\end{lstlisting}

There is one interesting issue in the tip design. Many builders taper the
leading and trailing edges so the tip is lighter. In this design, I decided to
taper just the leading and trailing edge tip spars and make the circular arc
and the tip thinner square stock. The puzzle is how to do this in \osc.

We saw how to taper a spare in our introduction to \osc. To get a double
tapered spar, we just add another trimming block.  Figure
\ref{fig:wing-tip.png} shows the final tip section.

\importimage{wing-tip.png}{Wing Tip}

All we need to do now, is generate the opposite tip. We could duplicate all of
the code we just set up, bu there is an easier way. \osc\ has an interesting
operation called {\bf mirror} that will create a mirror image of something. I
used this feature to create the opposite tip.

All that remains is to add the dihedral. This is done by rotating the completed
tip by the correct angle. If you specified the tip height, you need to calculate
that angle using the {\bf atan2} function we saw earlier.


\subsubsection{Tip Templates}

I kept this design simple on purpose. The only templates needed will be used
for forming the ribs and the curved segments at the tips. Fortunately, \osc\ can help
generate printed templates we can use! Here is the code used.

\includecode{demo/printer-test.scad}

The {\it scale} operation is used to scale the shape so it prints properly.
Basically, we set up code that generates a tip section, then flatten it back
into a 2D drawing. The operation that does this is called a {\it projection}.
Once that is set up, we will ask \osc\ to export this model as an SVG graphics
file which can then be printed. I use my Chrome browser to do the printing on my
home printer.  Figure \ref{fig:tip-template.png} shows what the template looks
like in Chrome.

\importimage{tip-template.png}{Tip Template}

\subsubsection{Wing Assembly}

All we need to to to complete the wing is generate the center section, then
generate the two wing tips and rotate them into place at the proper dihedral angle.

Figure{fig:wing.png}{Final Wing} shows the completed wing.

\importimage{wing.png}{Wing Assembly}


