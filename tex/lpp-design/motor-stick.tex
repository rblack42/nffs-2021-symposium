\subsection{Motor Stick}

With the flying surfaces set up, we can now turn to the fuselage parts. The
most important of these is the motor stick, which supports everything else in
the model. The motor stick must also bear the forces imposed by the wound up
rubber motor that will power this craft.

We can set up the motor stick several ways, but I will use a {\it polygon} to
define the basic shape, then use {\it linear\_extrude} to generate the actual
object.

Here is the basic layout we will use:

\importtikzfigure{motor-stick}{Basic Motor Stick}

This design provides support for the front bearing and the rear hook. The
module only needs one parameter: the thickness of the stock you will be using
for this part.

\begin{lstlisting}
module motor_stick(thickness=1/8) {
  ...
}
\end{lstlisting}

The final shape is centered along the {\bf X axis} with the bottom of the stick
lying on that axis to make final assembly of the model easier.

\importimage{motor-stick.png}{motor-stick.png}

I added one new feature to this module, which will not show up in the printed
article. You can ask \osc\ to color shapes using the {\it color} command. For
this motor stick, I just added this code to the module:

\begin{lstlisting}

module motor_stick*thickness=1/8) {
  color(WOOD_Balsa)
    rotate[(90,0,0)])
      translate([0,0,thickness/2])
  ...
}
\end{lstlisting}

I also added a single line at the top of the file:

\begin{lstlisting}
include <colors.scad>
\end{lstlisting}

The {\bf colors.scad} file is one I found online. I added my own color for balsa!

