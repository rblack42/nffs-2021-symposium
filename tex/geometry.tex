\section{Constructive Solid Geometry}

OpenSCAD builds 3D models using a small set of primitive shapes, and a set of
movement and combining operations to create more complex models.

\subsection{Primitive Shapes}

Openscad supports both 2D and 3D shapes. We will be using some simple 2D
shapes, like circles and rectangles, and more complex 2D shapes like a polygon.
The 3D shapes we will use include spheres, cylinders, and cubes. All of these
shapes can be scaled and moved around using simple movement operations.

\subsection{3D Primitives}
For our first look at how you do things in OpenSCAD, here is a piece of code
that will show the three basic 3D shapes:

\includecode{demo/demo1.scad}

Figure \ref{fig:demo1.png} shows the result.

\importimage{demo1.png}{Demo 1}

Each primitive shape is created at the origin. Cubes are created in the region
where all three coordinates are positive. Spheres are created with the center
of the sphere at the origin. The cylinder is centered along the {\bf Z} axis.
If you look closely, you will see a small representation of the coordinate
directions at the lower left of this image.

We used a {\it translate} operation to move shapes aside so they do not
overlap. The numbers inside square brackets control the distance we want to
{\it translate} the following shape in the {\bf [x,y,z]} directions. This
bracketed group of numbers is is called a {\it vector} which we will use a lot
in our work.

Notice how I indent code to show how things happen. In this example, we {\it
translate} the following {\it cube} shape. The semicolon ends this command.
Failing to put semicolons where they are needed is a common mistake when
writing \osc\ code.

These shapes do not look quite right. The problem is that \osc\  generates
approximations to the rounded shapes, using a set of small polygons to build up
the model. If we make these polygons smaller, things look better. All we need
to do to fix this is change the code so it looks like this:

\includecode{demo/demo2.scad}

Figure \ref{fig:demo2.png} shows a much better result. The special variable
{\it \$fn} controls the resolution of rounded objects. Bigger numbers make
things look smoother but cost of longer times to generate images on the screen.


\importimage{demo2.png}{Demo 2}

Some shapes are smart and can form different versions of themselves:

\includecode{demo/demo3.scad}

Figure \ref{fig:demo3.png} shows a warped cube and cylinder. Spheres are not so
smart, they stay spheres unless we warp them with external commands.

\importimage{demo3.png}{Demo 3}


\subsection{2D primatives}

We will use a few 2D shapes in this design, including the circle and square,
which act much like their 3D counterparts. A more interesting 2D shape we will
use is the {\it polygon}.

\includecode{demo/polygon-demo1.scad}

Here, we create two {\it variables} and set them equal to a list of vectors. 2D
vectors have only 2 numbers, for the {\bf X} and {\bf Y} coordinate values. The
first list defines a set of six points: three for the outer triangle, and three
more for the inner triangle. The second list identifies {\it paths} meaning a
continuous line that makes up a closed circuit, one for the outer triangle, and
one for the inner triangle. The numbers refer to the position of vectors in the
first list (programmers count starting at zero!) That final {\bf 10} parameter
is not important here, it helps the operation work properly.

I know this is a bit confusing, but we will not need much of this kind of code
in our design work. Remember to try things and see what happens.

\importimage{polygon-demo1.png}{Polygon Demo 1}

Figure \ref{fig:polygon-demo1.png} shows a  2D shape with no thickness, although
\osc\ gives it enough of a thickness to show up on the screen.

We can use this 2D shape to create a 3D object by {\it extruding it} in the
{\bf Z} direction:

\includecode{demo/polygon-demo2.scad}

Figure \ref{fig:polygon-demo2.png} definitely shows an interesting shape. We
will use {\it extrusion} to make some parts that would be difficult to construct
with just the basic primitive shapes.

\importimage{polygon-demo2.png}{Polygon Demo 2}

Obviously, we can form some interesting things with \osc. But things get
even more interesting when we start combining multiple shapes to form more
complex objects.

\subsection{Movement Operations}

We saw the {\it translate} operation earlier. We can also {\it rotate} a shape.
In this command we provide a vector of angles (in degrees) that we want to use
to rotate the shape. Each number in the vector will be used to rotate the shape
around the coordinate axis associated with that number. For instance {\bf
rotate([90,0,0])} will rotate the shape around the {\bf X} axis ninety degrees.
This operation uses the {\it right-hand} rule. If you want to rotate around the
{bf X} axis, take your right hand and point the thumb in the direction of
increasing {\bf X} in your coordinate system. Your fingers ``curl'' around that
axis in a positive direction.

Combining translations and rotations is done by writing both commands like
this:

\begin{lstlisting}
  translate)[10,15,0])
    rotate[90,0,0])
      cube(1,1,5);
\end{lstlisting}

It helps to read this bottom up. We are creating a {\it cube} at the origin. We
rotate it so it is aligned the way we want, then we translate that result to
the position we have chosen. The semicolon at the end of this list ends the
command. Notice that I indent so show what I want my code to do.

Be warned that you can swap the {\it translate} and {\it rotate} commands, but
you might not get the result you expect. Rotations are applied to the shape as
it is positioned when the command is processed. If you rotate after
translating, The shape will swing a long way!

\subsection{Combining Operations}

We form more complex objects by moving things around and combining them to form
new objects. An example found in the {\it Wikipedia} article on CSG~\cite{csgwiki}
demonstrates these operations.

Suppose you wanted to build something that looks like Figure \ref{fig:csg-demo.png}.

\importimage{csg-demo.png}{CSG Example Shape}

We can form this shape using three cylinders, a sphere, and a cube. We use all
three basic combining operations to construct the final shape.

Here is the OpenSCAD code used to generate this shape:

\includecode{demo/csg-demo.scad}

There is a point to be made here. We can move two objects together so they
touch, like a rib to a spar, but we do not really need to join them together in
this design work. Visually, things will look right, but the two objects remain
separate. Joining them together to make a combined part would be important if we
were going to 3D print the object. Since we do not have he technology to print
with balsa (yet), I will not worry about combining the components of our design
to create a single airplane object.

\subsubsection{Modules}

\osc\ lets you package a number of operations in a {\it module} that you can
activate later, one or more times. In fact those primitive shapes were all
predefined {\it modules}. The module can have parameters, which makes this a
powerful way to manage shapes that are similar, but differ depending on the
parameters you specify.  We saw that when we showed ``warped'' shapes earlier.
We will create a basic rib module for this model, and use parameters to control
the exact rib we want.

All modules have a unique name in your code.  The name you choose should help
you remember what the module is all about. In this example, we are interested
in the final {\bf part} shape, which is constructed using the difference
operation. This final module uses two supporting modules to build the part. You
can write your code almost any way you like, but it is common to use
spaces,indentations, and newlines to organize your code to make reading it
easier.  Also, we surround a sequence of individual operations inside of curly
braces when needed. I always indent any code inside of these braces.

When you add parameters to a module, you define names for each one between the
parentheses. Commas separate parameters if you have more than one. You can
optionally provide a default value for each parameter by adding an equal sign
followed by the default value you want. When you activate the module, you must
provide actual values you want the module to use. You can just provide a
sequence of numbers in the right order with commas separating them, or you can
add the parameter name from the definition, an equal sign, then the new value
you want. In this case, the order is not important, and you can leave off any
parameters where you are happy with the default value.  The rules for all of
this are detailed in the \osc\ {\it User Manual} \cite{userman}, so I will
not go further in this discussion here.

\subsubsection{Building the Example Shape}

To build this part, we first set up three cylinders, aligned along each
coordinate axis. The {bf center} parameter, sets each cylinder up with the
origin of the coordinate system at the exact center of the cylinder. Remember, The {\bf
\$fn=32} parameter is really only needed to make the cylinders actually look
round.

Notice that all three of these cylinders occupy the same space. In the real
world, we could not do that, but in our 3d modeling world this is common. We form
the {\bf union} of these three overlapping cylinders to form one merged shape.

The outer shell of our part is made up of the {\bf intersection} of a sphere
and a cube. We size the cube shape so it trims off six sides of the sphere
where holes will end up.  Finally, we use the {\bf difference} operator to
carve out the inside of the part, using our three-cylinder shape.

Successfully building 3D models involves visualizing what you want, then
arranging simple shapes as needed and performing these three basic combining
operations to generate the gadget you want! It takes practice! The more you
experiment the better you will get!

I encourage you to fire up \osc\ and type in this code. You will be better
able to see how things work by doing this!
