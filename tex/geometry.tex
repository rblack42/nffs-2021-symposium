\section*{\mbox{Constructive Solid Geometry}}

OpenSCAD builds 3D models using a small set of primitive shapes, and a set of
movement and combining operations to create more complex models.

\subsection*{Primitive Shapes}

Openscad supports both 2D and 3D shapes. We will be using some simple 2D
shapes, like circles and rectangles, and more complex 2D shapes like a polygon.
The 3D shapes we will use include spheres, cylinders, and cubes. All of these
shapes can be scaled and moved around using simple movement operations.

\subsection*{3D Primitives}
For our first look at how you do things in OpenSCAD, here is a piece of code
that will show the three basic 3D shapes:

\includecode{scad/demo1.scad}

Figure \ref{fig:demo1.png} shows the result.

\importimage{demo1.png}{Demo 1}

Each primitive shape is created at the origin. Cubes are created in the region
where all three coordinates are positive. Spheres are created with the center
of the sphere at the origin. The cylinder is centered along the {\bf Z} axis.
If you look closely, you will see a small representation of the coordinate
directions at the lower left of this image.

These shapes do not look quite right. The problem is that OpenSCAD generates
approximations to the rounded shaped, using a set of small polygons to build up
the model. If we make these polygons smaller, things look better. All we need
to do to fix this is change the code so it looks like this:

\includecode{scad/demo2.scad}

Figure \ref{fig:demo2.png} shows a much better result.

\importimage{demo2.png}{Demo 2}

Some shapes are smart and can form different versions of themselves:

\includecode{scad/demo3.scad}

Figure \ref{fig:demo3.png} shows a warped cube and cylinder. Spheres are not so
smart, they stay spheres unless we warp them with external commands.

\importimage{demo3.png}{Demo 3}

Specifying things in 3D is often done using the notation seen in the {\it
cube()} command. The numbers are the size we want in the {\bf [x,y,z]}
directions. This is a {\it vector} which we will use a lot in our work.

\subsection*{2D primatives}

2D shapes include the circle and square, which act much like their 3D
counterparts. A more interesting 2D shape is the {\it polygon}.

\includecode{scad/polygon-demo1.scad}

Here, we create {\it variables} and set them equal to a list of vectors. 2D
vectors have only 2 numbers, for the {\bf X} and {\bf Y} coordinate values. The
first list defines a set of six points: three for the outer triangle, and three
more for the inner triangle. The second list identifies {\it paths} meaning a
continuous line that makes up a closed circuit, one for the outer triangle, and
one for the inner triangle. (I know this is confusing, but we will not need
much of this kind of code in our design work.)

\importimage{polygon-demo1.png}{Polygon Demo 1}

Figure \ref{fig:polygon-demo1.png}shows a  2D shape with no thickness, although
OpenSACD gives it enough of a thickness to show up on the screen.

We can use this 2D shape to create a 3D object by {\it extruding it} in the
{\bf Z} direction:

\includecode{scad/polygon-demo2.scad}

Figure \ref{fig:polygon-demo2.png} definitely shows an interesting shape.

\importimage{polygon-demo2.png}{Polygon Demo 2}

Obviously, we can form some interesting things with OpenSCAD> But things get
even more interesting when we start combining multiple shapes to form even more
complex objects.

\subsection{Combining Operations}

We form more complex objects by moving thngs around and combining them to form other objects. An example found in the {\it Wikipedia} article on CSG demonstrates these operations.

Suppose you wanted to build something that looks like Figure \ref{fig:csg-demo.png}.

\importimage{csg-demo.png}{CSG Example Shape}

This shape was inspired from an example in the Wikipedia article on CSG
\cite{csgwiki}.

We can form this shape using three cylinders, a sphere, and a cube. We use all
three basic combining operations to construct the final shape.

Here is the OpenSCAD code used to generate this shape:

\includecode{scad/csg-demo.scad}

There is a lot to absorb here, but taken a piece at a time, it is not so bad.

\subsubsection*{Modules}

OpenSCAD lets you package a number of operations in a {\it module} that you can
activate later, one or more times. The module can have parameters, which makes
this a powerful way to manage shapes that are similar, but differ in some set
of parameters. We will create a basic rib module for this model, and use
parameters to control the exact rib we want.

The name we choose for each module should help you remember what the module is
all about. In this example, we are interested in the final {\bf part} shape,
which is constructed using the difference operation. This final module uses two
supporting modules to build the part. Note that the code does not care about
how you write your code, but it is common to use indentation to show that
operations are being applied to one of more groups of operations.Also, we
surround a sequence of individual operations inside of curly braces when
needed.

To build this part, we first set up three cylinders, aligned along each
coordinate axis. The {bf center} parameter, sets each cylinder up with the
origin of the coordinate system at the exact center of the cylinder. That
strange {\bf \$fn=32} parameter is needed ot make OpenSCAD generate cylinders
that actually look round. Inside, OpenSCAD uses small straight line segments to
generate a curve, and by default the size of these lines is pretty large, so we
raise the number of segments we want with this number. Too many and rendering
our model will be slow.

Notice that all three of these cylinders occupy the same space. In the real
world, we could not do that, but in our modeling world this is common. We form
the {\bf union} of these three overlapping cylinders to form one merged shape.
Many times, we could skip the {\bf union} operation and just place the shapes where
we want. This would be an important issue if we were actually going to 3D print
an object, but it is not really important when all we want to do is see our
design.

The outer shell of our part is made up of the {\bf intersection} of a sphere
and a cube. We size the cube shape so it trims off the six sides of the sphere
where holes will end up.  Finally, we use the {\bf difference} operator to
carve out the inside of the part, using our three-cylinder shape.

Successfully building 3D models involves visualizing what you want, then
arranging simple shapes as needed and performing these simple combining
operations to generate the gadget you want! It takes practice! The more you
experiment the better you will get!

I encourage you to fire up OpenSCAD and type in this code. You will be better
able to see how things work by doing this!
