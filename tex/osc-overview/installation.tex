Installing \osc\ is pretty simple using instructions found on the project's
website. Once it is installed, open it up and take a look at the basic
interface:

\importimage{opening-screen}{OpenSCAd interface}

There are three areas we will be using in this view:

\begin{itemize}
\item{Editor - the left panel where you type in your code.}
\item{Preview - the top right panel will be where your model is displayed.}
\item{Messages - the bottom right panel is where error messages will be shown
when processing your code.}
\end{itemize}

I will not try to show everything you need to know about the language \osc\
uses for describing a model. Instead, I will show fragments of code to give you
a feel for what you need to write to design your model. The project
website~\cite{blackr} has more details, as does the \osc\ {\it User
Manual}~\cite{userman}.

I highly recommend that you  print out a copy of the \osc\
``cheat-sheet'' is available here:~\cite{osccheat}. It will be a good reference
as you see \osc\ code examples.

\subsection{Installing Python}

In the analysis part of this discussion, we will be using some \PY\ programs to
do some of our work. \PY\ is another free tool, available for all platforms. It
even comes pre-installed on some (sadly, not on PCs though). You should install
this tool if you wish to follow along with this design.

\PY\ programs are just text files. I like to use an editing tool designed for
programmers to create my code. I have used a program called {\bf gvim} for
years as a professional software developer, but many other tools are available.
All operating systems provide some form of editor that will do the job.

Finally, some of the concepts in this article require running programs from the
{\it command line}. Many of you have never seen this interface, although it is
available on all systems. Basically, instead of clicking on some icon with
your mouse to start up a program, you enter a line of text into the interface
and tell the operating system what you want to do. I will only show a little of
this  here, more details are on the project website.

Ready to get started? Let's look at \osc!


