\subsubsection{\osc\ Basics}

We will be writing some program code in this design work. The code is not that
complicated, at least for most things we will cover. However, if you are not a
programmer, there are some details you need to be aware of.

First, you will be creating and managing a bunch of code files, which are just
text files processed by one of our tools. I keep everything about a project in
a directory (OK, folder for you PC folks) and create sub directories as needed
to keep things organized. All of our tools will open up files on our system and
process them to produce something we need. I have watched far too many
beginners get confused by not keeping things organized and separate from other
work you do on your system!

\osc\ lets you write your code in one file or in multiple files. All \osc\ code
files have names ending with {\bf .scad}. \PY\ files all end with {\bf .py}.
You can see many such files by exploring the Internet. All of my code is kept
on a public server used by millions of developers, called GitHub. 3D projects
complete with the files needed to produce prints of those designs can be found on
Thingiverse \cite{thingy}. You can learn a lot by reading code you find on these sites.

I like to split up a design into multiple files in order to keep them short and
focused on one just part of the design. If you split things up you will need to
use either an {\it include} or a {\it use} line specifying the file you want to
access with the code in the present file. If you choose {\it include}, all that
code in the second file will be processed as though it had been typed in the
current file.  On the other hand, using the {\it use} line only makes the names
from the second file available in this one. The code in the second file will
only be processed when those names are encountered in the current file.

A typical setup is to create a single file with dimensions you want every piece
of code to be able to use. You {\it include} that file like so:

\begin{lstlisting}
include <math-magik-data.scad>
\end{lstlisting}

\subsubsection{The Design Process}

Building a 3D model is a trial and error process. You type in or modify your
code, then click on a command to process that code. You look at the preview
window to see your model, and search the message area for hints about what went
wrong.  Non-programmers will find this a bit frustrating, but this takes
practice to master, so do not get discouraged. My advice is to always take
small steps.  That limits the number of problems you face in getting things to
work.

The best way to learn anything new is to experiment. Beginning programmers are
always searching the Internet for solutions they can copy into their problems,
but the only thing you actually learn when copying and pasting stuff is how to
copy and paste. You will learn far more by typing in code yourself - at least
until you get more proficient at this. Reading the code gives you a chance to
really think about what is going on. There is nothing wrong with looking at
code written by others. Many times studying that code will teach you how to
better write your own code. I will show you enough code in this design to give
you a feel for how you do things using \osc. The actual code I generated for
this design is on the project Github account~\cite{blackr}.

I highly recommend building small files that generate one part of your overall
design. Test that component until you are sure it looks like what you want.
Then use that part in building other components. I like to work from small
parts up to bigger assemblies, and that is how we will work through this
design. Don't be afraid to fire up \osc\ and try things are you read this
article. Of course, you should look at the project website to see all the code
in greater detail.

Now we can start work on our design!

