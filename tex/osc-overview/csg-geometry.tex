What does that even mean?  Basically it means we construct bigger shapes using
small set of primitive shapes, and a set of movement and combining operations.

The 3D shapes we will use include spheres, cylinders, and cubes. All of these
shapes can be scaled and moved around using simple movement operations.  We
will also use some simple 2D shapes, like circles and rectangles, We will also
use a more complex 2D shape called a polygon.  2D shapes have no thickness, but
are useful when creating more complicated 3D shapes.

\subsubsection{3D Primitive Shapes}

For our first look at how you do things in \osc, here is a piece of code
that shows the three basic 3D shapes:

\includecode{demo/demo1.scad}

Figure \ref{fig:demo1.png} shows the result.

\importimage{demo1.png}{Demo 1}

Each primitive shape is created at the origin. Cubes are created in the region
where all three coordinates are positive. Spheres are created with the center
of the sphere at the origin. The cylinder is centered along the {\bf Z} axis.
If you look closely, you will see a small representation of the coordinate
directions at the lower left of this image.

We used a {\it translate} operation to move shapes aside so they do not
overlap. The numbers inside square brackets control the distance we want to
{\it translate} the following shape in the {\bf [x,y,z]} directions. This
bracketed group of numbers is is called a {\it vector} which we will use a lot
in our work. The numbers \osc\ uses internally are in millimeters, but that is
only important when we print things. I prefer to pretend those numbers are in our
more familiar inches, and scale things when we generate printed output.

Notice how I indent code to show how things happen. This is strictly my style.
\osc\ does not care about how you style your code. In this example, we {\it
translate} the following {\it sphere} shape, then translate the following {\it
cylinder} shape as well. The semicolon ends this command.
Failing to put semicolons where they are needed is a common mistake when
writing \osc\ code.

These shapes do not look quite right. The problem is that \osc\  generates
approximations to the rounded shapes, using a set of small polygons to build up
the model. If we make these polygons smaller, things look better. All we need
to do to fix this is change the code so it looks like this:

\includecode{demo/demo2.scad}

Figure \ref{fig:demo2.png} shows a much better result. The special variable
{\it \$fn} controls the resolution of rounded objects. Bigger numbers make
things look smoother but it will take a bit longer to generate images on the screen.

\importimage{demo2.png}{Demo 2}

Some shapes are smart and can form different versions of themselves:

\includecode{demo/demo3.scad}

Figure \ref{fig:demo3.png} shows a warped cube and cylinder. Spheres are not so
smart, they stay spheres unless we warp them with external commands.

\importimage{demo3.png}{Demo 3}


\subsubsection{2D primitive Shapes}

We will use a few 2D shapes in this design, including the circle and square,
which act much like their 3D counterparts. A more interesting 2D shape we will
use is the {\it polygon}.

\includecode{demo/polygon-demo1.scad}

Here, we create two {\it variables} and set them equal to a list of vectors. 2D
vectors have only 2 numbers, for the {\bf X} and {\bf Y} coordinate values. The
first list defines a set of six points: three for the outer triangle, and three
more for the inner triangle. The second list identifies {\it paths} meaning a
continuous line that makes up a closed circuit, one for the outer triangle, and
one for the inner triangle. The numbers refer to the position of vectors in the
first list (programmers count starting at zero!) That final {\bf complexity=10}
parameter is not important here, it helps the operation work properly.

I know this is a bit confusing, but we will not need much of this kind of code
in our design work. Remember to try things and see what happens.

\importimage{polygon-demo1.png}{Polygon Demo 1}

Figure \ref{fig:polygon-demo1.png} shows a  2D shape with no thickness, although
\osc\ gives it enough of a thickness to show up on the screen.

subsubsecton{Extrusions}

We can use this 2D shape to create a 3D object by {\it extruding it} in the
{\bf Z} direction. This operation ``pull'' the 2D shape upward to produce a 3D
shape!

\includecode{demo/polygon-demo2.scad}

Figure \ref{fig:polygon-demo2.png} definitely shows an interesting shape. We
will use {\it extrusion} to make some parts that would be difficult to construct
with just the basic primitive shapes.

\importimage{polygon-demo2.png}{Polygon Demo 2}

Obviously, we can form some interesting things with \osc.

\subsubsection{Movement Operations}

We saw the {\it translate} operation earlier. We can also {\it rotate} a shape.
In this command we provide a vector of angles (in degrees) that we want to use
to rotate the shape. Each number in the vector will be used to rotate the shape
around the coordinate axis associated with that number. For instance {\bf
rotate([90,0,0])} will rotate the shape around the {\bf X} axis ninety degrees.
This operation uses the {\it right-hand} rule. If you want to rotate around the
{bf X} axis, take your right hand and point the thumb in the direction of
increasing {\bf X} in your coordinate system. Your fingers ``curl'' around that
axis in a positive direction.

Combining translations and rotations is done by writing both commands like
this:

\begin{lstlisting}
  translate)[10,15,0])
    rotate[90,0,0])
      cube(1,1,5);
\end{lstlisting}

It helps to read this bottom up. We are creating a {\it cube} at the origin. We
rotate it so it is aligned the way we want, then we translate that result to
the position we have chosen. The semicolon at the end of this list ends the
command. Notice that I indent so show what I want my code to do.

Be warned that you can swap the {\it translate} and {\it rotate} commands, but
you might not get the result you expect. Rotations are applied to the shape as
it is positioned when the command is processed. If you rotate after
translating, The shape will swing a long way!

\subsubsection{Convenience Operations}

That code looks messy to non-programmers, so I set up a system that uses
terminology more familiar with modelers. Basically, I packed up normal \osc\
operations into a set of operations with names you might find easier to use.

These operations are designed so you are visualizing things from the pilot's
perspective.  Here are a few of those terms:

\begin{itemize}
  \item{{\bf slide\_right(y)} - move to the right}
  \item{{\bf slide\_left(y)} - move to the left}
  \item{{\bf slide\_forward(x)} - move forward}
  \item{{\bf slide\_backward(x)} - move backward}
  \item{{\bf slide\_up(z)} - move upward}
  \item{{\bf slide\_down(z)} - slide downward}
\end{itemize}

Other operations are provided, their function should be easy to figure out for
airplane geeks! I will be using these in the actual design code found on the
project website.

\subsubsection{Combining Operations}

In 3D modeling, we often want to combine shapes to create more complex shapes.
That is important if we want to 3D print something. However, for most of our
work in this design, we will not need to actually fuse two parts together, just
position them correctly. However, there is one important combining operation we
will use a lot. The {\bf difference} operation actually takes two shapes that
overlap, and slices off the part where they intersect.

\includecode{scad/spar-trim.scad}

Figure \ref{fig:trim-block.png}shows what this looks like.

\importimage{trim-block.png}{Spar Trimming block}


This example shows some nice features of \osc. The first line calculates the
angle we need to use to taper  the spar to half its height. The {\bf
difference} operation takes the first shape inside those curly braces, in this
case our basic spar, and uses the next shape, the trimming block properly
aligned, to slice away where they intersect. That funny sharp character in front
of the second cube makes \osc\ show you the trimming block for debugging.
Normally, the trimming block is invisible, and the intersection evaporates.
Like sanding blocks on steroids!

There are other combining operations we could use, but they will not show up
much in our design work. The {\bf union} operation combines two shapes into
one, and the {\bf intersection} operation takes two overlapping shapes an
produces only that area where they intersect. We will use that last one in our
analysis work in the last section.

\subsubsection{Modules}

\osc\ lets you package a number of operations in a {\it module} that you can
activate later, one or more times. In fact those primitive shapes were all
predefined {\it modules}. The module can have parameters, which makes this a
powerful way to manage shapes that are similar, but differ depending on the
parameters you specify.  We will create a basic rib module for this model, and
use parameters to control the exact rib we want.

All modules need to have a unique name in your code.  The name you choose should
help you remember what the module is all about. When you add parameters to a
module, you define names for each parameter as well. These are surrounded by
parentheses, with commas separating them if you have more than one. You can
optionally provide a default value for each parameter by adding an equal sign
followed by the default value you want. When you activate the module, you will
provide actual values you want the module to use. You can just provide a
sequence of numbers in the right order with commas separating them, or you can
add the parameter name from the definition, an equal sign, then the new value
you want. In this case, the order is not important, and you can leave off any
parameters where you are happy with the default value.  The rules for all of
this are detailed in the \osc\ {\it User Manual} \cite{userman}, so I will not
go further in this discussion here.

Successfully building 3D models involves visualizing what you want, then
arranging simple shapes as needed and performing these three basic combining
operations to generate the gadget you want! It takes practice! The more you
experiment the better you will get! I encourage you to fire up \osc\ and type
in this code. You will be better able to see how things work by doing this!

