Designing a new model airplane usually involves generating a plan of some sort,
then constructing a prototype model from that plan. Of course you can use a
pencil and paper to generate your plans, but if you think you might want to
publish the plan, you will need to use some form of {\it Computer Aided Design}
tool to produce your final plan. Unfortunately many popular CAD tools are
complex, and often too expensive for the average modeler.

Having recently retired from teaching Computer Science, and finally getting back into
model building, I decided to design a new indoor model for the {\it Limited
Pennyplane} class. As part of the design process, I wanted to see that airplane
in 3D even before I built the first prototype. I decided to use a different
form of CAD tool: OpenSCAD \cite{openscad}, a tool designed for computer
programmers!

While that description may discourage some folks from reading further, rest
assured that this particular tool is simple enough that non-programmers can
certainly master it. In fact, some teachers have successfully managed to get
elementary school kids to use OpenSCAD to design simple 3D models.

OpenSCAD is an open-source (meaning free) 3D modeling program, available on all
major platforms. It is commonly used by folks designing parts to be printed on
3D printers. What makes OpenScad different is how you generate the design.
Instead of using your mouse to drag things around on the screen, you describe
your model in a simple programming language. Formally, OpenSCAD uses something
called {\it Constructive Solid Geometry} to construct your model, then gives
you a visual interface you can use to examine your 3D model in detail.

I will only show example code from the project so you can get a feel for the
design process. You are encouraged to explore the project website for much
better documentation and complete source code. \cite{blackr}.

