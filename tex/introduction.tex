\section{Introduction}

In case you have not noticed, 3D printers are becoming as common as the X-Acto
knife in home shops these days. They let you, or your kids, build some amazing
parts, some of which find their way into our model airplanes. If you look
around on {\it Thingiverse} \cite{thingy} you will discover that many of the
files needed for 3D printing were generated using a neat open-source {\it
Computer Aided Design} tool called \osc. As a retired Computer
Science Professor, I have used this tool for a number of projects, and I
decided to see how it could help with my current indoor model building.

\osc is not your usual CAD tool. It is a programmers tool, meaning that you
write fairly simple program code describing your model, then generate a visual
representation of your design you can see on your computer screen. Once you are
happy with the design, \osc can export your design in file formats that could be
post-processed as part of the path to a 3D printer. Unfortunately, we probably are
not going to see competition ready 3D printed indoor models anytime soon, so we will
not explore 3D printing in this article.

Do not be frightened off by having to write program code for \osc. Like
anything new, it can be a bit intimidating for beginners, but I have worked on
making the code needed simple enough that even non-technical
folks should be able to generate useful results.

In this article I will show how I developed a new design for a {\it Limited
Pennyplane} model using \osc. I will explain the basic concepts used in \osc
to set up a 3D design. This is not a complete tutorial, but covers enough to
help you understand the rest of the article, Next, we will see how to visualize
your design and examine it in detail as a full 3D model. Finally, we will
analyze the design to calculate some useful properties, like estimated weight
and center of gravity locations. I hope to show you that even non-technical
folks can use this tool to help design new models, and explore new design ideas
along the way.

All of the code and more detailed documentation on this project is available on
the project website at {\bf https://github.com/rblack42/math-magik}
\cite{blackr}. I will try to limit computer programming jargon, and present
concepts in more familiar modeling terms.



