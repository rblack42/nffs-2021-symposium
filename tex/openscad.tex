\section{OpenSCAD}

Installing OpenSCAD is oretty simple using instructions found on the projects
website. Once it is installed, open it up and take a look at the basic
interface:

\importimage{opening-screen}{OpenSCAd interface}

There are three areas we will be using in this view:

\begin{itemize}
\item{Editor - the left panel where you type in your code.}
\item{Preview - the top right panel will be where your model is displayed.}
\item{Messages - the bottom right panel is where error messages will be shown
when processing your code.}
\end{itemize}

I will not try to show everything you need to know about the language {\it
OpenSCAD} uses for describing a model. Instead, I will show fragments of code
to give you a feel for what you need to write to design your model. The project
website~\cite{blackr} has more details, as does the {\it OpenSCAD} {\it User
Manual}~\cite{userman}.

A Handy ``cheat-sheet'' is available here:~\cite{osccheat}.

\subsection{Organizing Your Code}

\osc\ lets you write your code in one file or in multiple files. I like to
split up a design into multiple files in order to keep them short and focused
on one just part of the design. If you split things up you will need to use either
an {\it include} or a {\it use} line specifying the file you want to access with
the code in the present file. If you choose {\it include}, all that code in the
second file will be processed as though it had been typed in the current file.
On the other hand, using the {\it use} line only makes the names from the
second file available in this one. The code in the second file will only be
processed when those names are encountered in the current file.

A typical setup is to create a single file with variables you want every piece
of code to be able to use. You {\it include} that file like so:

\begin{lstlisting}
include <m th-magik-data.scad>
\end{lstlisting}

Another example might be in a file defining the wing for this design. That file
needs to use ribs, which I define in another file. In my wing file I would
add this line

\begin{lstlisting}
use <rib.scad>
\end{lstlisting}

You will see a lot of this notation in the full project code files on the
project website.

\subsection{The Designng Process}

Building a 3D model is a trial and error process. You type in or modify your
code, then click on a command to process that code. You look at the preview
window to see your model, and search the message area for hints about what went
wrong.  Non-programmers will find this a bit frustrating, but this takes
practice to master, so do not get discouraged. My advice is to always take
small steps.  That limits the number of problems you face in getting things to
work.

The best way to learn anything new is to experiment. Beginning programmers are
always searching the Internet for solutions they can copy into their problems,
but the only thing you actually learn when copying and pasting stuff is how to
copy and paste. You will learn far more by typing in code yourself - at least
until you get more proficient at this. Reading the code gives you a chance to
really think about what is going on. There is nothing wrong with looking at
code written by others. Many times studying that code will teach you how to
better write your own code. I will show you enough code in this design to give
you a feel for how you do things using \osc. The actual code I generated for
this design is on the project Github account~\cite{blackr}.

I highly recommend building small files that generate one part of your overall
design. Test that component until you are sure it looks like what you want.
Then use that part in building other components. I like to work from small
parts up to bigger assemblies, and that is how we will work through this
design. Don't be afraid to fire up \osc\ and try things are you read this
article. Of course, you should look at the project website to see all the code
in greater detail.

