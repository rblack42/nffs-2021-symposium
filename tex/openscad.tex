\section*{OpenSCAD}

Installing OpenSCAD is oretty simple using instructions found on the projects
website. Once it is installed, open it up and take a look at the basic
interface:

\importimage{opening-screen}{OpenSCAd interface}

There are three areas we will be using in this view:

\begin{itemize}
\item{Editor - the left panel where you type in your code.}
\item{Preview - the top right panel will be where your model is displayed.}
\item{Messages - the bottom right panel is where error messages will be shown
when processing your code.} 
\end{itemize}

Building a model is a trial and error process. You type in or modify your code,
then click on a command to process that code. You look at the preview window to
see your model, and search the message area for hints about what went wrong.
Non-programmers will find this a bit frustrating, but this takes practice to
master, so do not get discouraged. My advice is to always take small steps.
That limits the number of problems you face in getting things to work.

The best way to learn anything new is to experiment. Beginning programmers are
always searching the Internet for solutions to their problems, but the only
thing you actually learn when copying and pasting stuff is how to copy and
paste. You will learn far more by typing in code yourself - at least until you
get more proficient at this. There is nothing wrong with looking at code
written by others. Many times studying that code will teach you how to better
write your own code. I will show you enough code in this design to give you a feel
for how you do things using OpenSCAD. The actual code I generated for this
design is on the project Github account \cite{blackr}.
