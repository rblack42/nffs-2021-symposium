\section*{Flight Box}

Since indoor model airplanes are so extremely light and fragile, it is
important that you give thought to how you are gong to transport the model.
Most builders do not show up at a flying site with just one model. Often, they
bring several for each event they want to fly. Many serious flyers build a nice
carrying case for their models. I decided to design one for Math-Magik!

When I was first getting into indoor competitions back in my college days, I
met Bud Tenny, who was the editor of a now-classic series of newsletters called
{\it Indoor News and Views}. These newsletters provided a wealth of information on
building indoor airplanes, and provided tips on how to fly them. Often there
were sketches of simple boxes made out of cardboard, or foam board that looked
like possibilities. However, I also have a copy of Ron Williams {\it Building and
Flying Indoor Model Airplanes} \cite{williams} , and ran into the ECIM
case, originally designed by the {\it East Coast Indoor Modelers} group. The design
presented here is a version of that case:

\importimage{flight-box.png}{Flight Box}


\subsection{Basic Construction}

As seen in figure \ref{fig:flight-box.png}, the box is made up of three
sections. One holds a number of wings (four are shown here). One holds the
stabilizers, and the middle section holds fuselages and propellers. The two end
sections are hinged to the center section so it opens as seen above. Closed,
this design measures 22"w x 9"h x 16"d. The outer sections are 6 inches deep
and the center section ins 4 inches deep.

The sides are constructed out of 1/8" plywood and all corners are braced with
1/2" pine strips. The hinges are piano hinges that run the full height of the
edges.

\subsection{OpenSCAD Design}

Designing this box with OpenSCAD turns out to be a nice way to figure out how
parts of the airplane will be transported and stored. The design of the
airplane structure presented so far allows both the wing and stabilizer to be
removed and stored separately. By attaching the wing posts to the motor stick and
using paper tubes for the attachments to flying surfaces on top of those posts,
the wings are not cluttered with structure and can be stacked neatly. The
stabilizers also stack nicely.

Laying the body on its side lets those be stacked as well.

That leaves the propellers! They can be stacked separately in the space above
the fuselage sections,.

When all of these items are placed in the box design, there is still room for
another model. My current thinking is to adds an A-6 model or two to this box's
payload!

I hope you see how handy a tool like \osc\ can be when you want to create
things to support your hobby!
